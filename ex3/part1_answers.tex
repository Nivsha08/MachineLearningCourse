%% LyX 2.3.2 created this file.  For more info, see http://www.lyx.org/.
%% Do not edit unless you really know what you are doing.
\documentclass[oneside,english]{amsart}
\setlength{\parskip}{\medskipamount}
\setlength{\parindent}{0pt}
\usepackage{amsthm}
\usepackage{graphicx}
\usepackage{setspace}
\doublespacing

\makeatletter

%%%%%%%%%%%%%%%%%%%%%%%%%%%%%% LyX specific LaTeX commands.
%% A simple dot to overcome graphicx limitations
\newcommand{\lyxdot}{.}


%%%%%%%%%%%%%%%%%%%%%%%%%%%%%% Textclass specific LaTeX commands.
\numberwithin{equation}{section}
\numberwithin{figure}{section}
\newenvironment{lyxlist}[1]
	{\begin{list}{}
		{\settowidth{\labelwidth}{#1}
		 \setlength{\leftmargin}{\labelwidth}
		 \addtolength{\leftmargin}{\labelsep}
		 \renewcommand{\makelabel}[1]{##1\hfil}}}
	{\end{list}}

\makeatother

\usepackage{babel}
\begin{document}
\title{\doublespacing{}Machine Learning - Exercise 3}
\author{Anat Balzam, Niv Shani}
\maketitle
\begin{doublespace}

\section*{Part 1 - Probability Theory Questions}

\section*{\textbf{Question 1}}
\end{doublespace}

\begin{doublespace}
We define the following events:
\end{doublespace}
\begin{description}
\begin{doublespace}
\item [{A}] We randomly got a Goldstar from the bar.
\item [{B}] The box that was moved from the storage to the bar was a Goldstar
box.
\end{doublespace}
\end{description}
\begin{doublespace}
We need to calculate the probability that a Stella box was moved,
given the fact we randomly got a Goldstar at the bar.

If a box of Stella was moved to the bar, we got 6 more bottles of
Stella at the bar. Thus the probability to randomly get a Goldstar
changes:

\[
P(A)=\frac{15}{35+6}=\frac{15}{41}
\]

To calculate $P(A)$ we'll use the law of total probability:

\[
P(A)=P(B)P(A|B)+P(\overline{{B}})P(A|\overline{{B}})=\frac{4}{11}\cdot\frac{21}{41}+\frac{7}{11}\cdot\frac{15}{41}=0.419
\]

We need to calculate: $P(\overline{{B}}|A)$. Thus:

\[
P(\overline{{B}}|A)=\frac{{P(\overline{{B}}\cap A)}}{P(A)}=\frac{\frac{7}{11}\cdot\frac{15}{41}}{0.419}=\frac{5}{9}
\]

\end{doublespace}
\begin{doublespace}

\section*{\textbf{\newpage Question 2}}
\end{doublespace}
\begin{lyxlist}{00.00.0000}
\item [{(a)}] Define \textbf{A}: the probability that a ship will be detected.\\
Using the law of total probability we get:
\[
P(A)=0\cdot0.8+0.2\cdot0.7+0.3\cdot0.6+0.5\cdot0.5=0.57
\]
\item [{(b)}] Define \textbf{B}: the probability of a ship to be in zone
C. Thus: $P(B)=0.3$.\\
Using conditional probability we get:
\[
P(B|A)=\frac{P(B\cap A)}{P(A)}=\frac{0.3\cdot0.6}{0.57}=0.3157
\]
\item [{(c)}] Define \textbf{C}: the probability of a ship to be in zone
B. Thus: \textbf{$P(C)=0.2$}.\\
Using conditional probability we get:
\[
P(C|A)=\frac{P(C\cap A)}{P(A)}=\frac{0.2\cdot0.7}{0.57}=0.2456
\]
\end{lyxlist}
\begin{doublespace}

\section*{\textbf{\newpage Question 3}}
\end{doublespace}

\section*{\textbf{\newpage Question 4}}
\begin{lyxlist}{00.00.0000}
\item [{(a)}] Define a random variable \textbf{X}: the number of descent
meals at Karnaf during a specific week.\\
We will treat a week as a series of 5 independent experiences, with
$p=0.7$ to get a descent meal at Karnaf.\\
Thus:
\[
X\sim B(n=5,p=0.7)
\]
\\
From the binomal distribution we get:
\[
P(X=3)={5 \choose 3}\cdot0.7^{3}\cdot0.3^{2}=0.3087
\]
\item [{(b)}] Using the binomial distribution again, we get:
\[
\begin{aligned}P(X & \geq2)=\\
= & 1-[P(X=0)+P(X=1)]\\
= & 1-[{5 \choose 0}\cdot0.7^{0}\cdot0.3^{5}+{5 \choose 1}\cdot0.7^{1}\cdot0.3^{4}]\\
= & 1-[0.00243+0.02835]\\
= & 0.96922
\end{aligned}
\]
\item [{(c)}] Since there are $N\geq30$ independet samples, and the conditions
for the CLE theorem for the binomial distribution holds, we expect
the average to hold:
\[
\overline{{X_{n}}}=\frac{\sum_{i=1}^{100}x_{i}}{100}\sim B(np,np(1-p))
\]
\[
\Rightarrow\overline{{X_{n}}}\sim B(70,\sqrt{21})
\]
\\
Meaning we expect the average (the mean value) to be $\mu=70$
\end{lyxlist}

\section*{\textbf{\newpage Question 5}}
\begin{lyxlist}{00.00.0000}
\item [{(a)}] $\forall(x,y)\in D\cap C$, it holds:\\
1. $x\geq0$\\
2. $|x^{2}+y^{2}|\leq1$\\
Thus, we can calculate:
\begin{align*}
P((x,y) & \in D\cap C|(x,y)\in C)\\
= & \frac{P[((x,y)\in D\cap C)\cap((x,y)\in C)]}{P((x,y)\in C)}\\
= & \frac{P(x\geq0)\cdot P(|x^{2}+y^{2}\leq1|)\cdot P(x\geq0)}{P(x\geq0)}\\
= & P(x\geq0)\cdot P(|x^{2}+y^{2}\leq1|)\\
= & \frac{1}{2}\cdot\pi=\frac{\pi}{2}\\
\end{align*}
Thus, we can treat the question as a series of independent experiences,
with $p=\frac{\pi}{2}$ chance to success.\\
Meaning:
\[
X\sim B(50,\frac{\pi}{2})
\]
\item [{(b)}] The CDF of X from 1 to 50:\\
\includegraphics[width=0.8\linewidth]{\string"/Users/niv/Desktop/Screen Shot 2019-04-10 at 21.56.36\string".png}
\end{lyxlist}

\end{document}
